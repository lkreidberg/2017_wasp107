%%
%% Beginning of file 'sample61.tex'
%%
%% Modified 2016 September
%%
%% This is a sample manuscript marked up using the
%% AASTeX v6.1 LaTeX 2e macros.
%%
%% AASTeX is now based on Alexey Vikhlinin's emulateapj.cls 
%% (Copyright 2000-2015).  See the classfile for details.

%% AASTeX requires revtex4-1.cls (http://publish.aps.org/revtex4/) and
%% other external packages (latexsym, graphicx, amssymb, longtable, and epsf).
%% All of these external packages should already be present in the modern TeX 
%% distributions.  If not they can also be obtained at www.ctan.org.

%% The first piece of markup in an AASTeX v6.x document is the \documentclass
%% command. LaTeX will ignore any data that comes before this command. The 
%% documentclass can take an optional argument to modify the output style.
%% The command below calls the preprint style  which will produce a tightly 
%% typeset, one-column, single-spaced document.  It is the default and thus
%% does not need to be explicitly stated.
%%
%%
%% using aastex version 6.1
\documentclass[twocolumn]{aastex61}

%% The default is a single spaced, 10 point font, single spaced article.
%% There are 5 other style options available via an optional argument. They
%% can be envoked like this:
%%
%% \documentclass[argument]{aastex61}
%% 
%% where the arguement options are:
%%
%%  twocolumn   : two text columns, 10 point font, single spaced article.
%%                This is the most compact and represent the final published
%%                derived PDF copy of the accepted manuscript from the publisher
%%  manuscript  : one text column, 12 point font, double spaced article.
%%  preprint    : one text column, 12 point font, single spaced article.  
%%  preprint2   : two text columns, 12 point font, single spaced article.
%%  modern      : a stylish, single text column, 12 point font, article with
%% 		  wider left and right margins. This uses the Daniel
%% 		  Foreman-Mackey and David Hogg design.
%%
%% Note that you can submit to the AAS Journals in any of these 6 styles.
%%
%% There are other optional arguments one can envoke to allow other stylistic
%% actions. The available options are:
%%
%%  astrosymb    : Loads Astrosymb font and define \astrocommands. 
%%  tighten      : Makes baselineskip slightly smaller, only works with 
%%                 the twocolumn substyle.
%%  times        : uses times font instead of the default
%%  linenumbers  : turn on lineno package.
%%  trackchanges : required to see the revision mark up and print its output
%%  longauthor   : Do not use the more compressed footnote style (default) for 
%%                 the author/collaboration/affiliations. Instead print all
%%                 affiliation information after each name. Creates a much
%%                 long author list but may be desirable for short author papers
%%
%% these can be used in any combination, e.g.
%%
%% \documentclass[twocolumn,linenumbers,trackchanges]{aastex61}

%% AASTeX v6.* now includes \hyperref support. While we have built in specific
%% defaults into the classfile you can manually override them with the
%% \hypersetup command. For example,
%%
%%\hypersetup{linkcolor=red,citecolor=green,filecolor=cyan,urlcolor=magenta}
%%
%% will change the color of the internal links to red, the links to the
%% bibliography to green, the file links to cyan, and the external links to
%% magenta. Additional information on \hyperref options can be found here:
%% https://www.tug.org/applications/hyperref/manual.html#x1-40003

%% If you want to create your own macros, you can do so
%% using \newcommand. Your macros should appear before
%% the \begin{document} command.
%%
\newcommand{\vdag}{(v)^\dagger}
\newcommand\aastex{AAS\TeX}
\newcommand\latex{La\TeX}

%% Reintroduced the \received and \accepted commands from AASTeX v5.2
%\received{July 1, 2016}
%\revised{September 27, 2016}
%\accepted{\today}
%% Command to document which AAS Journal the manuscript was submitted to.
%% Adds "Submitted to " the arguement.
\submitjournal{ApJL}

%% Mark up commands to limit the number of authors on the front page.
%% Note that in AASTeX v6.1 a \collaboration call (see below) counts as
%% an author in this case.
%
%\AuthorCollaborationLimit=3
%
%% Will only show Schwarz, Muench and "the AAS Journals Data Scientist 
%% collaboration" on the front page of this example manuscript.
%%
%% Note that all of the author will be shown in the published article.
%% This feature is meant to be used prior to acceptance to make the
%% front end of a long author article more manageable. Please do not use
%% this functionality for manuscripts with less than 20 authors. Conversely,
%% please do use this when the number of authors exceeds 40.
%%
%% Use \allauthors at the manuscript end to show the full author list.
%% This command should only be used with \AuthorCollaborationLimit is used.

%% The following command can be used to set the latex table counters.  It
%% is needed in this document because it uses a mix of latex tabular and
%% AASTeX deluxetables.  In general it should not be needed.
%\setcounter{table}{1}

%%%%%%%%%%%%%%%%%%%%%%%%%%%%%%%%%%%%%%%%%%%%%%%%%%%%%%%%%%%%%%%%%%%%%%%%%%%%%%%%
%%
%% The following section outlines numerous optional output that
%% can be displayed in the front matter or as running meta-data.
%%
%% If you wish, you may supply running head information, although
%% this information may be modified by the editorial offices.
\shorttitle{\aastex\ Atmosphere of WASP-107b}
\shortauthors{Kreidberg et al.}
%%
%% You can add a light gray and diagonal water-mark to the first page 
%% with this command:
% \watermark{text}
%% where "text", e.g. DRAFT, is the text to appear.  If the text is 
%% long you can control the water-mark size with:
%  \setwatermarkfontsize{dimension}
%% where dimension is any recognized LaTeX dimension, e.g. pt, in, etc.
%%
%%%%%%%%%%%%%%%%%%%%%%%%%%%%%%%%%%%%%%%%%%%%%%%%%%%%%%%%%%%%%%%%%%%%%%%%%%%%%%%%

%% This is the end of the preamble.  Indicate the beginning of the
%% manuscript itself with \begin{document}.

\begin{document}

\title{Water, High-Altitude Condensates, and No Methane in the Atmosphere of the Warm Neptune WASP-107b}

%% LaTeX will automatically break titles if they run longer than
%% one line. However, you may use \\ to force a line break if
%% you desire. In v6.1 you can include a footnote in the title.

%% A significant change from earlier AASTEX versions is in the structure for 
%% calling author and affilations. The change was necessary to implement 
%% autoindexing of affilations which prior was a manual process that could 
%% easily be tedious in large author manuscripts.
%%
%% The \author command is the same as before except it now takes an optional
%% arguement which is the 16 digit ORCID. The syntax is:
%% \author[xxxx-xxxx-xxxx-xxxx]{Author Name}
%%
%% This will hyperlink the author name to the author's ORCID page. Note that
%% during compilation, LaTeX will do some limited checking of the format of
%% the ID to make sure it is valid.
%%
%% Use \affiliation for affiliation information. The old \affil is now aliased
%% to \affiliation. AASTeX v6.1 will automatically index these in the header.
%% When a duplicate is found its index will be the same as its previous entry.
%%
%% Note that \altaffilmark and \altaffiltext have been removed and thus 
%% can not be used to document secondary affiliations. If they are used latex
%% will issue a specific error message and quit. Please use multiple 
%% \affiliation calls for to document more than one affiliation.
%%
%% The new \altaffiliation can be used to indicate some secondary information
%% such as fellowships. This command produces a non-numeric footnote that is
%% set away from the numeric \affiliation footnotes.  NOTE that if an
%% \altaffiliation command is used it must come BEFORE the \affiliation call,
%% right after the \author command, in order to place the footnotes in
%% the proper location.
%%
%% Use \email to set provide email addresses. Each \email will appear on its
%% own line so you can put multiple email address in one \email call. A new
%% \correspondingauthor command is available in V6.1 to identify the
%% corresponding author of the manuscript. It is the author's responsibility
%% to make sure this name is also in the author list.
%%
%% While authors can be grouped inside the same \author and \affiliation
%% commands it is better to have a single author for each. This allows for
%% one to exploit all the new benefits and should make book-keeping easier.
%%
%% If done correctly the peer review system will be able to
%% automatically put the author and affiliation information from the manuscript
%% and save the corresponding author the trouble of entering it by hand.

\correspondingauthor{Laura Kreidberg}
\email{laura.kreidberg@cfa.harvard.edu}

\author{Laura Kreidberg}
\affiliation{Harvard Society of Fellows\
78 Mt. Auburn St.\\
Cambridge, MA 02138, USA}
\affiliation{Harvard-Smithsonian Center for Astrophysics\
60 Garden St.\\
Cambridge, MA 02138}
\nocollaboration

\author{Michael Line}
\affiliation{Arizona State University}
\nocollaboration

\author{Caroline Morley}
\affiliation{Harvard-Smithsonian Center for Astrophysics}
\affiliation{Sagan Fellow}
\nocollaboration

\author{Kevin Stevenson}
\affiliation{Space Telescope Science Institute}
\nocollaboration


%% Note that the \and command from previous versions of AASTeX is now
%% depreciated in this version as it is no longer necessary. AASTeX 
%% automatically takes care of all commas and "and"s between authors names.

%% AASTeX 6.1 has the new \collaboration and \nocollaboration commands to
%% provide the collaboration status of a group of authors. These commands 
%% can be used either before or after the list of corresponding authors. The
%% argument for \collaboration is the collaboration identifier. Authors are
%% encouraged to surround collaboration identifiers with ()s. The 
%% \nocollaboration command takes no argument and exists to indicate that
%% the nearby authors are not part of surrounding collaborations.

%% Mark off the abstract in the ``abstract'' environment. 
\begin{abstract}

WASP-107b is baller.

\end{abstract}

%% Keywords should appear after the \end{abstract} command. 
%% See the online documentation for the full list of available subject
%% keywords and the rules for their use.
\keywords{planets and satellites: individual (WASP-107b), planets and satellites: atmospheres}

%% From the front matter, we move on to the body of the paper.
%% Sections are demarcated by \section and \subsection, respectively.
%% Observe the use of the LaTeX \label
%% command after the \subsection to give a symbolic KEY to the
%% subsection for cross-referencing in a \ref command.
%% You can use LaTeX's \ref and \label commands to keep track of
%% cross-references to sections, equations, tables, and figures.
%% That way, if you change the order of any elements, LaTeX will
%% automatically renumber them.

%% We recommend that authors also use the natbib \citep
%% and \citet commands to identify citations.  The citations are
%% tied to the reference list via symbolic KEYs. The KEY corresponds
%% to the KEY in the \bibitem in the reference list below. 

%Figures and Tables:
% 1.  transit light curves
% 2.  best fit spectrum
% 3.  retrieval results
% 4.  table of transit depths
% 5.  JWST predictions for cloud models 


\section{Introduction} \label{sec:intro}

\section{Observations}
We observed a single transit of WASP-107b with HST's Wide Field Camera 3 (WFC3) instrument on UT 5-6 June 2017.  The transit observation consisted of five HST orbits. At the beginning of each 96-minute orbit, we took an image of the target with the F130N filter (exposure time = 4.2 s). This direct image is used for wavelength calibration. For the remainder of the target visibility period (about 45 minutes), we obtained time series spectra with the G141 grism, which provides low-resolution spectroscopy over the wavelength range $1.1 - 1.7\,\mu$m.  We used the NSAMP=6, SPARS\_25 readout mode (exposure time = 112 s) to optimize the efficiency of the observations, as determined by the \texttt{PandExo\_HST} planning tool\footnote{https://github.com/spacetelescope/PandExo\_HST}.  As is standard for observations of bright targets, we used the spatial scanning observing mode, which smears the spectrum in the spatial direction over the course of an exposure. The scan rate was FIXME arcseconds/sec.


\section{Data Reduction}
We reduced the data with the custom pipeline described in FIXME, which we summarize briefly here. For each exposure, we extracted the spectrum from each up-the-ramp sample (or ``stripe") separately using the optimal extraction algorithm of \citep{horne86}. The stripe spectra were then summed to create the final spectrum. For each stripe, the extraction box was 80 pixels high and centered on the stripe's midpoint in the spatial direction. To correct the change in dispersion solution over the length of the spatial scan, we interpolated each row to the wavelength scale of the row corresponding to the spectral trace. We also corrected for slight drift in the spectral direction (FIXME pixels) by interpolating each the first exposure.  To subtract the background, we took the median of sky pixels that were uncontaminated by the target spectrum (rows $5-250$, columns $5-15$). The typical background counts were low: 40 photoelectrons/pixel, in comparison to $3\times10^4$ photoelectrons/pixel in the stellar spectrum.

\section{Analysis}
The data analysis had two parts: the band-integrated ``white" light curve fit and the spectroscopic light curve fits.

\subsection{White Light Curve Fit}
To create the raw white light curve, we summed each spectrum over the FIXME pixels in the spectral trace.  The morphology of the white light curve is typical for WFC3 observations \cite{zhou17}: the flux increases increases asymptotically over each orbit (the ``ramp" effect) and there is a visit-long linear trend. The largest ramp occurs in the initial orbit (orbit zero), so we only fit data from orbits one through four in our analysis, following common practice.  We fit the light curve with the analytic model of the form $F(t) = S(t)\times T(t)$, where $S$ is a systematics model and $T$ is a transit model. We used the same systematics model as FIXME, equation FIXME.  To model the transit, we used the \texttt{batman} package \citep{kreidberg15b}.  The transit model parameters are the orbital period $p$, time of FIXME $t_c$, transit depth $r_p/r_s$, ratio of semi-major axis to stellar radius $a/r_s$, orbital inclination $i$, and the quadratic stellar limb darkening parameters $u_1$ and $u_2$.

In our initial fit, we allowed all the transit parameters to vary freely. We noticed two odd results: first, there was a slight increase in flux during orbit three, and second, the transit parameters were inconsistent with the precise analysis of the Kepler light curve from \cite{dai17}.  We conjectured that the anomalies could be due to a star-spot crossing in orbit three. WASP-107 is an active star and star spot crossings have been observed before FIXME. To test this idea, we gave the data in orbit three no weight in the fit. In this case, the transit parameters agree well with those of \cite{dai17} and the spot-crossing feature is more pronounced (see Figure FIXME). The amplitude of the spot crossing feature is FIXME ppm, consistent with a spot of FIXME size and temperature (cf dai17 FIXME). We also found that the fitted limb darkening parameters were conistent with predictions from a PHOENIX model of FIXME properties, calculated with the FIXME package.  FIXME: we tested interpolating models, didn't matter. 

In our final fit, we fixed the transit parameters $a/r_s$, $i$, $p$ on the values from \cite{dai17}.  We also fixed the limb darkening parameters on the PHOENIX model prediction. The root-mean-square residuals were FIXME ppm (not including orbit three), compared to the expected shot noise of FIXME ppm. The achieved rms is close to the best value obtained for WFC3 data (FIXME), and we attribute the slight excess noise to either additional star spot crossings or loss of flux from the edge of the detector \citep[as has been seen for other bright targets;][]{FIXME}. We used the Markov chain Monte Carlo algorithm to estimate parameter uncertainties (as implemented in the the \texttt{emcee} package; FIXME). We scaled the per-point uncertainties by a factor of FIXME to achieve a reduced $\chi_2$ value of unity. The transit time was FIXME and the planet/star radius was FIXME.

\subsection{Spectroscopic Light Curve Fits}
We binned the spectrum into 20 spectrophotometric channels from FIXME to FIXME $\mu$m.  We explored two different methods for fitting these spectroscopic light curves.  First we used the \texttt{divide-white} technique, which assumes that the light curve systematics are nearly constant as a function of wavelength \citep{stevenson??, kreidberg14a}. For this approach, we fit a transit model times the systematics vector from the white light curve fit ($F(t)/T(t)$), rescaled by a factor $C_\lambda + V_\lambda t$. As for the white light curve, we fixed some of the transit parameters on the estimates from \cite{dai17} and fixed the limb darkening on the PHOENIX model. The remaining free parameters are $C_\lambda$, $V_\lambda$, and $r_p/r_s$.  As a test, we fit for a linear limb darkening parameter and found the results were consistent with the PHOENIX model predictions. An advantage of this approach is that it automatically removes the star spot crossing, enabling us to use orbit three with no additional correction. We expect that this model introduces a small chromatic effect due to the changing spot crossing feature amplitude with wavelength, but this effect is below the level of noise in our data FIXME. The fits reach the photon noise and have a median reduced $\chi_2$ value of FIXME.  There are two channels with slightly higher

As a test, we also fit the spectroscopic light curves with the same analytic model we used for the white light curve. The results were nearly identical to those obtained with the \texttt{divide-white} technique, except offset by a constant value of FIXME ppm. We attribute the offset to the uncorrected star-spot crossing in orbit three. Although the spot is not detectable in the spectroscopic light curves due to their higher photon noise, it may still affect the fit. report the values from the 

explored different binning

\subsection{Independent Analysis}
We also performed an independent data reduction and fit using K. Stevenson's pipeline.


Confirmation from KBS

\section{Atmospheric Retrieval}
We use the CHIMERA chemically-consistent transmission retrieval tool described in \cite{kreidberg15} determine the basic atmospheric properties.   Briefly, the transmission spectrum solves the transmission geometry problem using the equations in \cite{brown11, tinetti12}.  We parametrize atmospheric composition with metallicity and carbon-to-oxygen ratio (C/O) under the assumption of thermochemical equilibrium using the NASA CEA routine \citep{gordon96} to compute the molecular abundances for H$_2$, He, H$_2$O, CH$_4$, CO, CO$_2$, NH$_3$, H$_2$S, Na, K, HCN, C$_2$H$_2$, TiO, VO, and FeH.    We have subsequently upgraded the transmission model to use correlated-K opacities \citep{lacis91, molliere15, amundsen16} for the aforementioned gases generated from the pre-tabulated line-by-line cross section database ($dv/v~10^6$) described in \cite{freedman14}. The transmission forward model is coupled with the powerful PyMultiNest tool \citep{buchner16} to solve the parameter estimation and model selection problems.  

The nominal model set-up requires a temperature-pressure profile (parameterized via the \citealt{guillot10} relations), the atmospheric metallicity and carbon-to-oxygen ratio for molecular composition determination, disquilibrium properties crudely parameterized with nitrogen species and carbon species quench pressures \citep{morley17}, and aerosol properties.  We experiment with three atmospheric scenarios within this setup (Table XX):  Scenarios with and without methane, and with/without patchy cloud cover (as described and implemented in \citealt{line16}.  In all three scenarios we fix the T-P profile "shape" but scale the irradiation temperature, which incorporates unknown variables like the albedo and heat transport efficiency.  T-P profile shape information is not readily retrievable, nor does it strongly influence information derived from WFC3 transmission spectra \citep[e.g.][]{kreidberg15}. We also switch off the quench pressure parameters as they are unlikely to strongly influence the equilibrium abundance profiles of the spectrally prominent, nearly constant-with-altitude abundance, species like CH$_4$, H$_2$O, and NH$_3$ over this temperature range \citep[e.g.][]{line11, moses13}.  Aerosols are simply parameterized with a ``gray cloud top pressure" with the option for cloud patchiness. We do not consider ``sloped hazes" at this time as they are unlikely to influence the results.

\section{Discussion} \label{sec:discuss}

\subsection{Methane}

\subsection{Comparative planetology with HAT-P-11b, HAT-P-26b}

\subsection{Constraints on Condensate Properties}
The amplitude of WASP-107b's spectral features is about half that expected for an atmosphere free of condensates. In our retrieval analysis, we explored two simple parameterizations to model condensates: a gray opacity source and a sloped haze.  The retrieved cloud-top pressure is FIXME at $1\,\sigma$ confidence.

We also considered physically motivated, self-consistent cloud and haze models based on .  To model WASP-107b's spectrum with self-consistent aerosols, we use the methods described in \citep{fortney08, morley15}. We include models from solar to $50\times$ solar metallicity and solar C/O ratio. We model clouds that form in cool atmospheres (Na$_2$S, KCl, ZnS, see \citealt{morley12}), varying the cloud sedimentation efficiency from 0.1 to 3. None of the cloudy models are sufficiently low amplitude to match the observed muted signal. We also model an ad hoc photochemical 'soot' layer near the top of the atmosphere, scaling results from previous photochemical models for GJ 436b \citep{line11, morley17}. With a sufficiently thick photochemical haze with particle sizes around $0.03-0.1$ microns, the amplitude of the model water feature matches that of the observations. 



\subsection{Atmospheric Metallicity Predictions from Interior Structure Modeling}


\section{Future Plans}
Combine data from Spitzer/HST
JWST GTO


\acknowledgments
Fei Dai
Jessica Spake

L.R.K. acknowledges support from the Harvard Society of Fellows and the Harvard Astronomy Department Institute for Theory and Computation.

\vspace{5mm}
\facilities{HST(WFC3)}

%% Similar to \facility{}, there is the optional \software command to allow 
%% authors a place to specify which programs were used during the creation of 
%% the manusscript. Authors should list each code and include either a
%% citation or url to the code inside ()s when available.

\software{astropy \citep{2013A&A...558A..33A},  
          Cloudy \citep{2013RMxAA..49..137F}, 
          SExtractor \citep{1996A&AS..117..393B}
          }

%% Appendix material should be preceded with a single \appendix command.
%% There should be a \section command for each appendix. Mark appendix
%% subsections with the same markup you use in the main body of the paper.

%% Each Appendix (indicated with \section) will be lettered A, B, C, etc.
%% The equation counter will reset when it encounters the \appendix
%% command and will number appendix equations (A1), (A2), etc. The
%% Figure and Table counter will not reset.

%\appendix


%% The reference list follows the main body and any appendices.
%% Use LaTeX's thebibliography environment to mark up your reference list.
%% Note \begin{thebibliography} is followed by an empty set of
%% curly braces.  If you forget this, LaTeX will generate the error
%% "Perhaps a missing \item?".
%%
%% thebibliography produces citations in the text using \bibitem-\cite
%% cross-referencing. Each reference is preceded by a
%% \bibitem command that defines in curly braces the KEY that corresponds
%% to the KEY in the \cite commands (see the first section above).
%% Make sure that you provide a unique KEY for every \bibitem or else the
%% paper will not LaTeX. The square brackets should contain
%% the citation text that LaTeX will insert in
%% place of the \cite commands.

%% We have used macros to produce journal name abbreviations.
%% \aastex provides a number of these for the more frequently-cited journals.
%% See the Author Guide for a list of them.

%% Note that the style of the \bibitem labels (in []) is slightly
%% different from previous examples.  The natbib system solves a host
%% of citation expression problems, but it is necessary to clearly
%% delimit the year from the author name used in the citation.
%% See the natbib documentation for more details and options.

\bibliographystyle{apj}
\bibliography{ms.bib}

%% This command is needed to show the entire author+affilation list when
%% the collaboration and author truncation commands are used.  It has to
%% go at the end of the manuscript.
%\allauthors

%% Include this line if you are using the \added, \replaced, \deleted
%% commands to see a summary list of all changes at the end of the article.
%\listofchanges

\end{document}

% End of file `sample61.tex'.
